\section{Installation}

This section is designed to serve as a step-by-step instruction of how to build and run QES-Winds. In the first part, packages required to build the code will be mentioned along with the oldest version of each package that satisfies the purpose. The next part will be interaction with the repository on GitHub in which the code is been stored to clone the code. Also, commands required for cloning the repository and building the executable of code, will be mentioned. The last part of this section will cover a brief description of how to change the input files of the code and run it.

\subsection{Required Packages}

The very first package needed to be installed is ''git'' package. It provides the ability to interact with GitHub and use commands to clone the repository, switch between different branches and etc. This package does not have any dependencies, so it is always recommended to install the latest version. 

\begin{itemize}
\item The next package inline is ''CMake'' and its GUI version ''CCMake''. It finds all the packages required, links them together and creates the ''makefile'' for building the code. CMake should be any version greater than 3.10.
\item QES-Winds also needs ''boost'' libraries in order to have access to C++ source libraries. Boost 1.66.0 is sufficient for the purpose of QES-Winds. 
\item ''Gdal'' libraries are necessary to read in Digital Elevation Models (DEM) and shapefile (for buildings). Version 2.3.1 of gdal libraries will do the job for our applications. 

\item The last library that needs to be installed is ''netcdf-c'' libraries along with netcdf interface with C++, version 4.6 is required. Netcdf libraries are essential for reading in WRF output files and writing QES-Winds results in netcdf format.

\item Finally, since QES-Winds is written in C++ and CUDA, ''gcc'' and ''CUDA'' compilers needed to be installed. Because there is a compatibility issue between versions of CUDA, gcc and Operating System(OS) (for more information go to https://docs.nvidia.com/cuda/cuda-installation-guide-linux/index.html), version of gcc that is compatible with the version of CUDA and OS is required. For CUDA, at least version 8.0 needs to be installed.
\end{itemize}

\subsection{Cloning QES-Winds from GitHub}

After making sure all the required packages are installed and ready to use, a copy of QES-Winds needs to be downloaded on the local computer (cloning process). To clone the code, go to the directory you want to have the code downloaded, open a terminal and type ''git clone [address to the repository]''. To get the address to the repository, go to the repository GitHub page, UtahEFD/QES-Winds-Public, click on the green button ''Code'' and copy the ''HTTPS'' address. It downloads a copy of the code in the “master” branch of the repository in your local directory.

\subsection{Building QES}

Next steps are:
\begin{itemize}
    \item Go to the folder created with name QES-Winds: ''cd QES-Winds-Public''.
    \item Create a build directory: ''mkdir build'' or ''sudo mkdir build''.
    \item Go to folder build: ''cd build''.
    \item Type: ''cmake ..''.
\end{itemize}

There is a chance that cmake fails to find all the packages needed for running the code (packages installed on unconventional directories). In this case, you need to do cmake with appropriate flags that point to those packages.

\begin{itemize}
    \item After cmake is done successfully, type: ''build''
    \item A successful build will result in creating the executable named ''qesWinds''
\end{itemize}
\subsection{Build Types}

The code support several build types: \textit{Debug}, \textit{Release}, \textit{RelWithDebInfo}, \textit{MinSizeRel}. You can select the build type
\begin{verbatim}
cmake -DCMAKE_BUILD_TYPE=Release ..
\end{verbatim}
\textit{Release} is recommanded for production

cmake options:
\begin{itemize}
    \item build code without CUDA support automatically if CUDA is not supported by the system
	\item build code with openmp support for future multithread application,openmp is not automatically enabled. if openmp support is enable (\verb|-DENABLE_OPENMP=ON|) the code will run a new red/black solver on the CPU. Unfortunately thread safety issues with some plume classes did not allow for an easy parallelization of the plume advection.
	\item default is \textit{RELEASE} with most warning off, \verb|-O3| optimization.
a dev mode was added \verb|-DENABLE_DEV_MODE=ON| showing warnings, will build the code in DEBUG this option is slow and recommended only for active development.
	\item ClangTidy option was added
	\item use \verb|-DENABLE_TESTS=ON| to enable unit, sanity, and regressions tests using Catch2 (https://github.com/catchorg/Catch2) 
\end{itemize}



\subsubsection{Linux}
On a general Linux system, such as Ubuntu 18.04 or 20.04, the following packages need to be installed:
\begin{itemize}
\item libgdal-dev
\item libnetcdf-c++4-dev
\item libnetcdf-cxx-legacy-dev
\item libnetcdf-dev
\item netcdf-bin
\item libboost-all-dev
\item cmake
\item cmake-curses-gui
\end{itemize}

If the system uses \verb|apt|, the packages can be installed using the following command:
\begin{verbatim}
apt install libgdal-dev libnetcdf-c++4-dev  libnetcdf-cxx-legacy-dev libnetcdf-dev netcdf-bin libboost-all-dev cmake cmake-curses-gui
\end{verbatim}


We separate the build 
\begin{verbatim}
mkdir build
cd build
cmake ..
\end{verbatim}
You can then build the source:
\begin{verbatim}
make
\end{verbatim}

\subsubsection{macOS}

The packages can be installed using Homebrew (https://brew.sh)

\begin{verbatim}
brew install cmake boost gdal hdf5 netcdf netcdf-cxx
\end{verbatim}

If openMP multithreading is desired: 
\begin{verbatim}
brew install libomp
\end{verbatim}

On intel silicon machines:
\begin{verbatim}
cmake -DNETCDF_LIBRARIES_CXX=/usr/local/lib/libnetcdf-cxx4.dylib -DENABLE_OPENMP=ON -DCMAKE_PREFIX_PATH=/usr/local/Cellar/libomp/15.0.7/ ..
\end{verbatim}

On apple silicon machines:
\begin{verbatim}
cmake -DCMAKE_PREFIX_PATH=/opt/homebrew/Cellar/libomp/15.0.7 -DENABLE_OPENMP=ON -DNETCDF_LIBRARIES_CXX=/opt/homebrew/lib/libnetcdf-cxx4.dylib ..
\end{verbatim}

\subsubsection{University of Utah - CHPC}

\textit{This is the preferred build setup on CHPC}

The code does run on the CHPC cluster. You need to make sure the correct set of modules are loaded.  Currently, we have tested a few configurations that use
\begin{itemize}
\item GCC 5.4.0 and CUDA 8.0
\item CCC 8.1.0 and CUDA 10.1 (10.2)
\item GCC 8.5.0 and CUDA 11.4
\end{itemize}
If you build with OptiX support, you will need to use CUDA 10.2 or newer configuration. Any builds (with or without OptiX) with CUDA 11.4 are preferred if you don't know which to use. Older configurations are provided in \verb|CHPC/oldBuilds.md|.

After logging into your CHPC account, you will need to load specific modules. In the following sections, we outline the modules that need to be loaded along with the various cmake command-line calls that specify the exact locations of module installs on the CHPC system.  

To build with GCC 8.5.0, CUDA 11.4, and OptiX 7.1.0 on CHPC. Please use the following modules:
\begin{verbatim}
module load cuda/11.4
module load cmake/3.21.4
module load gcc/8.5.0
module load boost/1.77.0
module load intel-oneapi-mpi/2021.4.0
module load gdal/3.3.3
module load netcdf-c/4.8.1
module load netcdf-cxx/4.2
\end{verbatim}
Or use the provided load script, which will always load the latest tested configuration.
\begin{verbatim}
source CHPC/loadmodules_QES.sh
\end{verbatim}
After completing the above module loads, the following modules are reported from `module list`:
\begin{verbatim}
Currently Loaded Modules:
  1) cuda/11.4    (g)   3) gcc/8.5.0      5) intel-oneapi-mpi/2021.4.0   7) netcdf-c/4.8.1
  2) cmake/3.21.4       4) boost/1.77.0   6) gdal/3.3.3                  8) netcdf-cxx/4.2
\end{verbatim}
After the modules are loaded, you can create the Makefiles with cmake.  We keep our builds separate from the source and contain our builds within their own folders.  For example, 
\begin{verbatim}
mkdir build
cd build
cmake -DCUDA_TOOLKIT_DIR=/uufs/chpc.utah.edu/sys/installdir/cuda/11.4.0 -DCUDA_SDK_ROOT_DIR=/uufs/chpc.utah.edu/sys/installdir/cuda/11.4.0 -DOptiX_INSTALL_DIR=/uufs/chpc.utah.edu/sys/installdir/optix/7.1.0 -DCMAKE_C_COMPILER=gcc -DNETCDF_CXX_DIR=/uufs/chpc.utah.edu/sys/installdir/netcdf-cxx/4.3.0-5.4.0g/include ..
\end{verbatim}
Upon completion of the above commands, you can go about editing and building mostly as normal, and issue the `make` command in your build folder to compile the source.

After you've created the Makefiles with the cmake commands above, the code can be compiled on CHPC:
\begin{verbatim}
make
\end{verbatim}
Note you \textit{may} need to type make a second time due to a build bug, especially on the CUDA 8.0 build.
